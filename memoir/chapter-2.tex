\chapter{Conceptos basicos}
\label{cha:conceptos_basicos}



En en caso real, la planta generadora proporciona los valores de potencia que se introducen en cada modelos. En este trabajo trataremos las plantas generadoras reales como modelos de Simulink. Los modelos serán aplicados con los resultados obtenidos de estas simulaciones. El objetivo de los modelos es predecir la potencia producida por la planta fotovoltaica. 

\section{Predicciñon energética sobre pronostico} 
\label{sec:prediccion_energetica_sobre_pronostico}

El gran problema de las energias renovables es su falta de estabilidad. El el caso de la energia solar, no habra dos dias que tengan la misma produccion energetica, aun que si se podrá exprimir la periodicidad del sol. 
Con los paneles fotovoltaicos, la radiación y la temperatura aunque son los valores con mas peso, no son los unicos necesarios para determinar con exactitud el rendimiento de un panel (angulo del sol, contaminacion, cobertura, tipo de panel, etc).
Por esto, se dedicara el trabajo a las predicciones post procesado de la energia. i.e. Usando como fuente de datos para los modelos de prediccion la potencia que genera una planta fotovoltaica.
Estos serán contrastados con los valores medidos (Extraidos de la simulación) del periodo predecido.
Si por ejemplo tenemos Abril y predecimos Mayo, contrastaremos los datos de Mayo (Que también tenemos) con su predicción. 	


\section{Solar} 
\label{sec:solar}

La prediccion energetica solar esta muy avanzada a dia de hoy, existen modelos teoricos, tanto probados en entornos cerrados como casos reales. 
En este proyecto nos centraremos en llevar a la practica modelos teoricos


\subsection{Modelos de predicción} 
\label{sub:modelos_de_prediccion}

Los modelos se expresan en terminos de energia y tienen horizontes de predicción diferentes (horarios principalmente). Por esto, para unificar todos los resultados se usará un horizonte de predicción de 24h.

En los modelos de predicción que trabajen los instantes anteriores (horizontes de horas) se...


\subsubsection{ Exponentially weighted moving average} 
\label{ssub:subsubsection_name}

Este modelo es el mas sencillo y tradicional. Para poder ajustarse al ciclo solar, se realizan subsegmentos de tamanyo \textbf{numero de muestras por dia}. 
Este se basa en una suma basculada de los valores Real y Pronosticado, siendo este ultimo el resultado del pronostico del dia anterior.

\begin{align}
	E_{PRED}(d+1) &= \rho E_{REAL}(d) + (1-\rho) E_{PRED}(d)
\end{align}

La energia pronosticada en un momento t y dia d sera la suma de la energia medida en el momento t del dia anterior y la energia pronosticada en el momento t del dia anterior.

En esencia, el algoritmo usa el historico de los dias anteriores para predecir el siguiente. Si la constante $\rho$ esta mas cerca del 1, el pronostico tendera a ser el valor del dia anterior. Sin embargo, si $\rho$ es mas proxima al 0, se explotara el registro acumulado. 

\hl{
Pros: Bajo procesamiento y consumo de memoria
Cons: Prediccion muy simple sin control de errores
}


\subsubsection{ Predictor for adaptive management developed at ETHZ} 
\label{ssub:subsubsection_name}

Este modelo de prediccion 

\begin{align}
	E_{S}(t) &= min( (1-\gamma) E_{N}(t) + (\gamma) R_{t}, E_N(t) ) \\
	R(t) &= \beta R(t-1) + (1-\beta) E_S(t-1) \\
	E_N(t) &= \alpha E_N(t-1) + (1-\alpha) E_S(t-1)
\end{align}

Siendo Es~ la energia producida por el modelo siendo ,
Es es la energia real producida por el pv,
En es la energia promedio medida en T ultimas muestras con factor $\alpha$
Finalmente R es el promedio a corto plazo con factor $\beta$

Este algoritmo se basa en el mismo principio que el EWMA separando el corto plazo del largo plazo.
\hl{
Pros: Tiene mejor factor de prediccion a corto plazo que el EWMA
Cons: Sigue sin filtrar variaciones habituales y no es inmune a imprevistos
}


\subsubsection{ Optimal 2D linar prediction filter} 
\label{ssub:subsubsection_name}

El fundamento de este algoritmo es la representacion matricial de la energia. Siendo las filas los dias y las columnas los momentos (horas en el estudio) ambas ordenadas en orden creciente, un nuevo dato de energia es igual a los valores de la columna (K dias anteriores) mas los valores de fila (K horas anteriores) mas los valores previos $((K-1)^2$ horas anteriores en dias anteriores).

Siendo el tamanyo de venta K y el nuevo valor $(K,K)$, los valores $(\_,K-1)$ son los dias, $(K-1,\_)$ son las horas y $([1,K-1],[1,K-1])$ son los valores previos no relacionados. 

Cada uno de estos 3 elementos sera multiplicado por unos coeficientes a1, a2 y a3. 

Este seria un ejemplo para tamanyo de ventana K = 2

\begin{align}
	X'_{i+1,j+1} &= a_1 X_{i,j} + a_2 X_{i,j+1} + a_3 X_{i+1,j}
\end{align}


\subsubsection{ Neural network prediction} 
\label{ssub:subsubsection_name}

Consideraremos tambien una red neuronal. Estas al estas destinadas a resolver problemas cuyas entradas estan fuera de lo esperado hacen aparentemente idonea la solucion ya que la variabilidad de las entradas (el clima) depende de muchas variables dificiles de contemplar simultaneamente.
Aprovecharemos la capacidad adaptativa y analitica con propagacion hacia atras para predecir valores con bajo coste de memoria.

La red neuronal estara disenyada con 2 capas de 8 neuronas y funcion sigmoidal $O = \frac{1}{(1+e^{-n})}$

El entrenamiento sera con el algoritmo de regularizacion Bayesiana, que mejorara el rendimiento con ruido, y con propagacion hacia atras, asi que las salidas conectaran con las salidas.




\subsubsection{ Weather Conditioned Moving Average} 
\label{ssub:subsubsection_name}

El algoritmo mas explotado en la literatura, WCMA, tiene sus bases en el EWMA con el valor anyadido de que tiene una vision global del periodo y control de cambios 

\begin{align}
	E_{(d,n+1)} &= \alpha E_{(d,n)} + (1-\alpha) \frac  {\sum_{i=1}^D E_{i,n+1}} {D} \\
	v^{d,n}_k &= E_{(d,n-k+1)} \frac {D}{\sum^D_i=1} E_{i,n-k+1} \\
	p_k &= 1-\frac {K-k+1}{K} \\
	GAP_k^{d,n} &= V^{d,n}(K) \times \big(P(K) / \sum P(K) \big)^T \\
	E^{GAP}_{d,n+1} &= \alpha E_(d,n) + (1-\alpha) GAP_K^{d,n} \frac{\sum^D_{i=1} E_{i,n+1}}{D}
\end{align}



\subsubsection{ Weather Conditioned Moving Average with Phase Displacement Regulator} 
\label{ssub:subsubsection_name}


\begin{align}
	PE(d,i) &= \delta PE(d-1,i) + (1-\delta) \frac  {\sum_{i=1}^D PE_{i,n+1}} {D} \\
	PDR_{d,n} &= \bigg[ \frac{\sum^{w-1}_{w_i = 0} [ PE(n-w_i) + PE(n+w_i) ] \gamma^{(w_i + 1)}}{\sum_{w_i = 0}^{w - 1} \gamma^{(w_i + 1)}} \bigg] \\
	E^{PDR}_(d,n+1) &= \alpha E_{d,n} + (1-\alpha ) GAP^{d,n}_K \frac{\sum^D_{d,n} E_{i,n+1} }{D} + PDR_{d,n+1}
\end{align}


\subsubsection{n4sid}
\label{ssub:n4sid}

Este algoritmo (Sistema de identificación de subespacios de espacios de estados) permite estimar un modelo de espacio de estados usando subespacios.

Un modelo de espacio de estados es un modelo que usa variables de estado para describir un sistema con un sistema de ecuaciones diferenciales de primer orden o ecuaciones diferenciales. 

Los modelos de espacio de estados son apropiados para realizar estimaciones (predicciones en este caso) ya que solo requieren del orden, el cual esta relacionado con el retardo de las entradas y salidas a usar en la ecuación diferencial.

La representación del modelo estimado es un sistema como:

\begin{align}
	x(t) = Ax(t) + Bu(t) + Ke(t) \\
	y(t) = Cx(t) + Du(t) + d(t)
\end{align}

Donde A,B,C y D son matrices del espacio de estados, K es la matriz de perturbaciones, $u(t)$ es la entrada, $y(t)$ es la salida, $e(t)$ es la perturbación y $x(t)$ es el vector de nx estados, con nx igual al orden.



\subsection{Conclusiones} 
\label{sub:conclusiones}

A nivel de calidad de predicción cualquiera de las opciones que exploten tanto el historico a largo plazo (Dias) como a corto plazo (momentos anteriores) y tenga en cuenta el peso de la proximidad de los valores, es unaa buena opción de predictor de serie temporal.

De estos modelos se puede extraer que los más favorables son el EWMA y el EWMA con PDR. Queda excluida la red neuronal debido a su excesivo tiempo de entrenamiento y predicción. 




\section{Eolica} 
\label{sec:eolica}

La predicción de energía eólica es mucho mas compleja que la solar ya que la irradiación a lo largo del día sigue una distribución normal. Luego se verá afectada por multiples factores como nubes o polución, pero el sol siempre sigue el mismo patrón. El viento, en cambio, no. Y esto dificulta mucho la tarea.

Abordaremos la predicción de energía eólica de 3 maneras:
\begin{itemize}
	\item Modelo ARIMA
	\item Red neuronal artificial
	\item n4sid
\end{itemize}

\subsection{Coneptos basicos} 
\label{sub:coneptos_basicos}

\subsubsection{ARIMA}
\label{ssub:arima}

ARIMA: AutoRegressive Integrated Moving Average. Este modelo se usa, o para predecir valores (Su proposito aquí) o bien para entender los valores pasados.

Para entender un poco como funcionan los modelos ARIMA lo explicaremosp por componentes.

El factor AutoRegressive, Auto regresivo o el AR de ARIMA define la variable en cualquier instante del tiempo como una combinación de los valores anteriores más un error.

El factor Moving Average, Media móvil o el MA de ARIMA define un valor de la variable como una suma de un valor $\alpha$ mas la suma ponderada de los errores.

El factor Integration o la I de ARIMA indica que un valor de la variable es la diferencia entre un valor y su sucesor, pudiendo haber sido realizada la diferencia varias veces.

Sabiendo esto el modelo ARIMA se enuncia como ARIMA(p,d,q) siendo p,d y q numeros naturales que representan la distancia del retardo a usar, el numero de veces que se realizó la resta en la diferencia y el orden de la ecuación de media movil.


\subsubsection{n4sid}
\label{ssub:n4sid}

Ver N4SID de predicción solar.



\subsection{Conclusiones} 
\label{sub:conclusiones}



\section{Solar vs Eolica} 
\label{sec:solar_vs_eolica}




