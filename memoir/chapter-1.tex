\chapter{Proposito y objetivos}
\label{cha:introduccion}


\section{Introducción}

Es bien sabido que el ser humano, desde tiempos inmemoriables, ha querido predecir el futuro. Y aunque en este trabajo sea solo en materia energética, seguiremos intentandolo. Dependiendo de la fuente puede ser más o menos factible e.g. Calcular la energía producida por una central geotérmica o una planta solar es mas factible que la producida por un molino de viento, pues de la fuente de calor o del sol se pueden extraer patrones y la velocidad del cambio es menor. Sin embargo, el viento (La segunda fuente que trataremos) tiene una gran variabilidad y no hay periodicidad aunque se puedan encontrar patrones.

Por esto estudiaremos modelos de predicción que tengan como objetivo predecir la energía que será producida por una planta solar o molino de viento. Con ello, y quedando fuera del ámbito de este trabajo, se podrán crean planes de estimación de consumo combinando la predicción, una fuente de energía auxiliar como unas baterias y la red eléctrica a fin de mejorar el consumo.


\section{Propósito}

Nuestro objetivo es examinar qué metodología tiene los mejores resultados observados en la predicción energética de una planta solar y una eolica con un horizonte de 24h y con un entrenamiento a lo largo de 14 dias.

