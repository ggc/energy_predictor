\documentclass[a4paper,11pt]{book}
\usepackage[spanish, activeacute]{babel}
\usepackage[utf8]{inputenc}
\usepackage{lmodern}
\usepackage[T1]{fontenc}
\usepackage{times}
\usepackage{graphicx}
\usepackage{multirow}
\usepackage{color,soul}
\definecolor{gray97}{gray}{.97}
\definecolor{gray75}{gray}{.75}
\definecolor{gray45}{gray}{.45}
\usepackage[colorlinks=true, linkcolor=black, urlcolor=blue, filecolor=blue]{hyperref}
\usepackage{amsfonts}
\usepackage{anysize}
\usepackage{algorithm}
\usepackage{algorithmic}
\usepackage{amsmath}
\input{spanishAlgorithmic}

\marginsize{3cm}{3cm}{2cm}{2cm}

\usepackage{listings}
\lstset{ frame=Ltb,
framerule=0pt,
aboveskip=0.5cm,
framextopmargin=3pt,
framexbottommargin=3pt,
framexleftmargin=0.4cm,
framesep=0pt,
rulesep=.4pt,
backgroundcolor=\color{gray97},
rulesepcolor=\color{black},
%
stringstyle=\ttfamily,
showstringspaces = false,
basicstyle=\small\ttfamily,
commentstyle=\color{gray45},
keywordstyle=\bfseries,
%
numbers=left,
numbersep=15pt,
numberstyle=\tiny,
numberfirstline = false,
breaklines=true,
%
tabsize=4
}

\newcommand\mytitle{Desarrollo de modelos de predicción de la generación de energía eólica y solar para la optimización de Centros de Datos en un entorno de Smart Grid}
\newcommand\myauthor{Gabriel Galán Casillas}
\newcommand\myadvisorA{José Luis Ayala Rodrigo}
\newcommand\myadvisorB{Marina Zapater Sancho}
\newcommand\mydegree{Ingeniería Informatica}
\newcommand\myyear{2017}
\newcommand\mycourse{2016/2017}

\title{Desarrollo de modelos de predicción de la generación de energía eólica y solar para la optimización de Centros de Datos en un entorno de Smart Grid}
\author{\myauthor}
\date{\today}
\makeindex
%\usepackage{graphicx}
\begin{document}
\baselineskip 1.5em     % 2em doble espacio, 1em espacio sencillo.

%%% PORTADA %%%
\titlepage{}
\setlength{\unitlength}{1 cm} %Especificar unidad de trabajo

\begin{center}
\textbf{{\LARGE Universidad Complutense de Madrid}\\[0.5cm]}
\textbf{
{\Large Facultad de Informática}}\\[0.5cm]
\begin{figure}[!ht]
\begin{center}
\includegraphics[height=5cm,]{images/logoucm.png}\\[1cm]
\end{center}
\end{figure}


\textbf{\LARGE \mytitle}\\[4cm]
\begin{tabular}{r l}
\Large Alumno	& \Large \textbf{\myauthor}	\\[0.3cm]
\Large Directores de proyecto	& \Large  \textbf{\myadvisorA} \\	
                            & \Large  \textbf{\myadvisorB}
\end{tabular}\\[4cm]
{\large Trabajo de Fin de Grado}\\[0.3cm]
{\large \mydegree}\\[0.3cm]
{\large Junio de \myyear}
\end{center}



%%% ÍNDICE %%%
\tableofcontents

%%%Resumen y palabras clave %%%
\chapter*{Palabras clave}
\addcontentsline{toc}{chapter}{Palabras clave}
\markboth{Palabras clave}{Palabras clave}

\section*{Palabras clave en Español}
\begin{itemize}
\item Modelado y simulación
\item Sistemas de eventos discretos
\item Formalismo DEVS y xDEVS
\item Raspberry Pi
\item Módulos del kernel de Linux
\item Circuitos lógicos y secuenciales
\end{itemize}

\section*{Keywords in English}
\begin{itemize}
\item Modeling and simulation
\item Discrete events system
\item DEVS formalism and xDEVS
\item Raspberry Pi
\item Linux kernel Modules
\item Logic and sequential circuits
\end{itemize}

% Index
\chapter*{Resumen}
\addcontentsline{toc}{chapter}{Resumen}
\markboth{Resumen}{Resumen}

\section*{Resumen en Español}
Resumen en español de la propuesta.

\section*{Summary in English}
Summary in english.


% Intro & obj
\chapter{Proposito y objetivos} % (fold)
\label{cha:energia_solar}

% chapter energia_solar (end)

\section{intro}
Sección 1 del capítulo 1

\subsection{sub de la intro} % (fold)
\label{sub:introduccion}

\section{proposito}

% subsection introduccion (end)

% Study
\chapter{Conceptos basicos}
\label{cha:conceptos_basicos}



En en caso real, la planta generadora proporciona los valores de potencia que se introducen en cada modelos. En este trabajo trataremos las plantas generadoras reales como modelos de Simulink. Los modelos serán aplicados con los resultados obtenidos de estas simulaciones. El objetivo de los modelos es predecir la potencia producida por la planta fotovoltaica. 

\section{Predicciñon energética sobre pronostico} 
\label{sec:prediccion_energetica_sobre_pronostico}

El gran problema de las energias renovables es su falta de estabilidad. El el caso de la energia solar, no habra dos dias que tengan la misma produccion energetica, aun que si se podrá exprimir la periodicidad del sol. 
Con los paneles fotovoltaicos, la radiación y la temperatura aunque son los valores con mas peso, no son los unicos necesarios para determinar con exactitud el rendimiento de un panel (angulo del sol, contaminacion, cobertura, tipo de panel, etc).
Por esto, se dedicara el trabajo a las predicciones post procesado de la energia. i.e. Usando como fuente de datos para los modelos de prediccion la potencia que genera una planta fotovoltaica.
Estos serán contrastados con los valores medidos (Extraidos de la simulación) del periodo predecido.
Si por ejemplo tenemos Abril y predecimos Mayo, contrastaremos los datos de Mayo (Que también tenemos) con su predicción. 	


\section{Solar} 
\label{sec:solar}

La prediccion energetica solar esta muy avanzada a dia de hoy, existen modelos teoricos, tanto probados en entornos cerrados como casos reales. 
En este proyecto nos centraremos en llevar a la practica modelos teoricos


\subsection{Modelos de predicción} 
\label{sub:modelos_de_prediccion}

Los modelos se expresan en terminos de energia y tienen horizontes de predicción diferentes (horarios principalmente). Por esto, para unificar todos los resultados se usará un horizonte de predicción de 24h.

En los modelos de predicción que trabajen los instantes anteriores (horizontes de horas) se...


\subsubsection{ Exponentially weighted moving average} 
\label{ssub:subsubsection_name}

Este modelo es el mas sencillo y tradicional. Para poder ajustarse al ciclo solar, se realizan subsegmentos de tamanyo \textbf{numero de muestras por dia}. 
Este se basa en una suma basculada de los valores Real y Pronosticado, siendo este ultimo el resultado del pronostico del dia anterior.

\begin{align}
	E_{PRED}(d+1) &= \rho E_{REAL}(d) + (1-\rho) E_{PRED}(d)
\end{align}

La energia pronosticada en un momento t y dia d sera la suma de la energia medida en el momento t del dia anterior y la energia pronosticada en el momento t del dia anterior.

En esencia, el algoritmo usa el historico de los dias anteriores para predecir el siguiente. Si la constante $\rho$ esta mas cerca del 1, el pronostico tendera a ser el valor del dia anterior. Sin embargo, si $\rho$ es mas proxima al 0, se explotara el registro acumulado. 

\hl{
Pros: Bajo procesamiento y consumo de memoria
Cons: Prediccion muy simple sin control de errores
}


\subsubsection{ Predictor for adaptive management developed at ETHZ} 
\label{ssub:subsubsection_name}

Este modelo de prediccion 

\begin{align}
	E_{S}(t) &= min( (1-\gamma) E_{N}(t) + (\gamma) R_{t}, E_N(t) ) \\
	R(t) &= \beta R(t-1) + (1-\beta) E_S(t-1) \\
	E_N(t) &= \alpha E_N(t-1) + (1-\alpha) E_S(t-1)
\end{align}

Siendo Es~ la energia producida por el modelo siendo ,
Es es la energia real producida por el pv,
En es la energia promedio medida en T ultimas muestras con factor $\alpha$
Finalmente R es el promedio a corto plazo con factor $\beta$

Este algoritmo se basa en el mismo principio que el EWMA separando el corto plazo del largo plazo.
\hl{
Pros: Tiene mejor factor de prediccion a corto plazo que el EWMA
Cons: Sigue sin filtrar variaciones habituales y no es inmune a imprevistos
}


\subsubsection{ Optimal 2D linar prediction filter} 
\label{ssub:subsubsection_name}

El fundamento de este algoritmo es la representacion matricial de la energia. Siendo las filas los dias y las columnas los momentos (horas en el estudio) ambas ordenadas en orden creciente, un nuevo dato de energia es igual a los valores de la columna (K dias anteriores) mas los valores de fila (K horas anteriores) mas los valores previos $((K-1)^2$ horas anteriores en dias anteriores).

Siendo el tamanyo de venta K y el nuevo valor $(K,K)$, los valores $(\_,K-1)$ son los dias, $(K-1,\_)$ son las horas y $([1,K-1],[1,K-1])$ son los valores previos no relacionados. 

Cada uno de estos 3 elementos sera multiplicado por unos coeficientes a1, a2 y a3. 

Este seria un ejemplo para tamanyo de ventana K = 2

\begin{align}
	X'_{i+1,j+1} &= a_1 X_{i,j} + a_2 X_{i,j+1} + a_3 X_{i+1,j}
\end{align}


\subsubsection{ Neural network prediction} 
\label{ssub:subsubsection_name}

Consideraremos tambien una red neuronal. Estas al estas destinadas a resolver problemas cuyas entradas estan fuera de lo esperado hacen aparentemente idonea la solucion ya que la variabilidad de las entradas (el clima) depende de muchas variables dificiles de contemplar simultaneamente.
Aprovecharemos la capacidad adaptativa y analitica con propagacion hacia atras para predecir valores con bajo coste de memoria.

La red neuronal estara disenyada con 2 capas de 8 neuronas y funcion sigmoidal $O = \frac{1}{(1+e^{-n})}$

El entrenamiento sera con el algoritmo de regularizacion Bayesiana, que mejorara el rendimiento con ruido, y con propagacion hacia atras, asi que las salidas conectaran con las salidas.




\subsubsection{ Weather Conditioned Moving Average} 
\label{ssub:subsubsection_name}

El algoritmo mas explotado en la literatura, WCMA, tiene sus bases en el EWMA con el valor anyadido de que tiene una vision global del periodo y control de cambios 

\begin{align}
	E_{(d,n+1)} &= \alpha E_{(d,n)} + (1-\alpha) \frac  {\sum_{i=1}^D E_{i,n+1}} {D} \\
	v^{d,n}_k &= E_{(d,n-k+1)} \frac {D}{\sum^D_i=1} E_{i,n-k+1} \\
	p_k &= 1-\frac {K-k+1}{K} \\
	GAP_k^{d,n} &= V^{d,n}(K) \times \big(P(K) / \sum P(K) \big)^T \\
	E^{GAP}_{d,n+1} &= \alpha E_(d,n) + (1-\alpha) GAP_K^{d,n} \frac{\sum^D_{i=1} E_{i,n+1}}{D}
\end{align}



\subsubsection{ Weather Conditioned Moving Average with Phase Displacement Regulator} 
\label{ssub:subsubsection_name}


\begin{align}
	PE(d,i) &= \delta PE(d-1,i) + (1-\delta) \frac  {\sum_{i=1}^D PE_{i,n+1}} {D} \\
	PDR_{d,n} &= \bigg[ \frac{\sum^{w-1}_{w_i = 0} [ PE(n-w_i) + PE(n+w_i) ] \gamma^{(w_i + 1)}}{\sum_{w_i = 0}^{w - 1} \gamma^{(w_i + 1)}} \bigg] \\
	E^{PDR}_(d,n+1) &= \alpha E_{d,n} + (1-\alpha ) GAP^{d,n}_K \frac{\sum^D_{d,n} E_{i,n+1} }{D} + PDR_{d,n+1}
\end{align}


\subsubsection{n4sid}
\label{ssub:n4sid}

Este algoritmo (Sistema de identificación de subespacios de espacios de estados) permite estimar un modelo de espacio de estados usando subespacios.

Un modelo de espacio de estados es un modelo que usa variables de estado para describir un sistema con un sistema de ecuaciones diferenciales de primer orden o ecuaciones diferenciales. 

Los modelos de espacio de estados son apropiados para realizar estimaciones (predicciones en este caso) ya que solo requieren del orden, el cual esta relacionado con el retardo de las entradas y salidas a usar en la ecuación diferencial.

La representación del modelo estimado es un sistema como:

\begin{align}
	x(t) = Ax(t) + Bu(t) + Ke(t) \\
	y(t) = Cx(t) + Du(t) + d(t)
\end{align}

Donde A,B,C y D son matrices del espacio de estados, K es la matriz de perturbaciones, $u(t)$ es la entrada, $y(t)$ es la salida, $e(t)$ es la perturbación y $x(t)$ es el vector de nx estados, con nx igual al orden.



\subsection{Conclusiones} 
\label{sub:conclusiones}

A nivel de calidad de predicción cualquiera de las opciones que exploten tanto el historico a largo plazo (Dias) como a corto plazo (momentos anteriores) y tenga en cuenta el peso de la proximidad de los valores, es unaa buena opción de predictor de serie temporal.

De estos modelos se puede extraer que los más favorables son el EWMA y el EWMA con PDR. Queda excluida la red neuronal debido a su excesivo tiempo de entrenamiento y predicción. 




\section{Eolica} 
\label{sec:eolica}

La predicción de energía eólica es mucho mas compleja que la solar ya que la irradiación a lo largo del día sigue una distribución normal. Luego se verá afectada por multiples factores como nubes o polución, pero el sol siempre sigue el mismo patrón. El viento, en cambio, no. Y esto dificulta mucho la tarea.

Abordaremos la predicción de energía eólica de 3 maneras:
\begin{itemize}
	\item Modelo ARIMA
	\item Red neuronal artificial
	\item n4sid
\end{itemize}

\subsection{Coneptos basicos} 
\label{sub:coneptos_basicos}

\subsubsection{ARIMA}
\label{ssub:arima}

ARIMA: AutoRegressive Integrated Moving Average. Este modelo se usa, o para predecir valores (Su proposito aquí) o bien para entender los valores pasados.

Para entender un poco como funcionan los modelos ARIMA lo explicaremosp por componentes.

El factor AutoRegressive, Auto regresivo o el AR de ARIMA define la variable en cualquier instante del tiempo como una combinación de los valores anteriores más un error.

El factor Moving Average, Media móvil o el MA de ARIMA define un valor de la variable como una suma de un valor $\alpha$ mas la suma ponderada de los errores.

El factor Integration o la I de ARIMA indica que un valor de la variable es la diferencia entre un valor y su sucesor, pudiendo haber sido realizada la diferencia varias veces.

Sabiendo esto el modelo ARIMA se enuncia como ARIMA(p,d,q) siendo p,d y q numeros naturales que representan la distancia del retardo a usar, el numero de veces que se realizó la resta en la diferencia y el orden de la ecuación de media movil.


\subsubsection{n4sid}
\label{ssub:n4sid}

Ver N4SID de predicción solar.



\subsection{Conclusiones} 
\label{sub:conclusiones}



\section{Solar vs Eolica} 
\label{sec:solar_vs_eolica}






% Practice
\chapter{Experimentación}

\graphicspath{ {./graphs/} }

En este capitulo explicaremos todo el proceso de exprerimentación, desde la recogida de datos hasta la obtención de resultados


\section{Entorno}
Para poder llevar a cabo unos resultados validos seria necesario tener una planta solar capaz de producir energía y la monitorización de esta. En este caso, dispondriamos de la potencia actual generada, la cual es la entrada a los algoritmos de predicción. 

Dado que no es así, simularemos este comportamiento con un proceso manual de recogida y procesado de datos compuesto por una planta simulada con actuación sobre un historico.
La temperatura y radiación sera recopilada por un api rest, almacenada en una base de datos, extraida y formateada para ser introducida en el modelo de planta fotovoltaica de matlab.
Tras ejecutar la simulación, los valores de potencia resultantes seran pasados a los distintos modelos para obtener las predicciones.
Una vez obtenidas las predicciones, seran contrastadas con los valores reales obteniendo asi graficas que prueben la certeza de estas.

A continuahción se explican cada uno de los componentes del entorno.

\begin{itemize}
    \item Preprocesado
    \begin{itemize}
        \item RESTful API de pronostico climatologico dark sky
        \item Servidor
        \item Parser
    \end{itemize}
    \item Planta simulada
    \begin{itemize}
        \item Base de datos
        \item Modelo de planta solar
    \end{itemize}
    \item Algoritmos de predicción
\end{itemize}


\subsection{RESTful API} 
\label{sub:API}
El primer paso para la puesta en marcha del modelo de planta solar es recoger los valores de radiación y temperatura.

Para la recogida de datos barajamos estas opciones:
\begin{enumerate}
    \item Estación meteorologia de AEMET con valores reales de su base de datos
    \item API publica
\end{enumerate}


\subsubsection{Estación meteorológica de AEMET}
\label{ssub:estación_meteorológica_de_aemet}

Fue nuetra primera opción por ser una base solida a la que tenemos acceso. Tiene puntos fuertes como proximidad, soporte técnico y fiabilidad.

Los datos eran proporcionados en el siguiente formato:

En la estación meteorológica de Barajas

\begin{table}[tb]
    \caption{caption here}
    \label{tab:tablename}
    \centering

    \begin{tabular}{l|cc}
    \hline

    \hline
    \textbf{column 1} & \textbf{column 2} & \textbf{column 3} \\
    \hline
        value1 & value2 & value3\\
    \hline

    \hline
    \end{tabular}
\end{table}

Indicativo: Indicativo climatológico
NOMBRE: Nombre estación
ALTITUD: Altitud de la estación (metros)
NOM\_PROV: Provincia
LONGITUD: Longitud geografica
(La ultima cifra indica la orientación: 1 para longitud E y 2 para W)
LATITUD: Latitud geografica
DATUM: Datum de referencia

TOTSOL: Insolación total diaria
PTJESOL: Porcentaje de insolación
SOL07: Insolación de 00 a 07
SOL13: Insolación de 07 a 13
SOL18: Insolación de 13 a 18
SOL00: Insolación de 18 a 24

Unidades y valores especiales:

Horas UTC (Tiempo Universal Coordinado)

Insolación en décimas de hora

Porcentaje de insolación en


En la estacion de Cuidad Universitaria y El Retiro

Campos incluidos:
Indicativo: Indicativo climatológico
NOMBRE: Nombre estación
ALTITUD: Altitud de la estación (metros)
NOM\_PROV: Provincia
LATITUD: Latitud geogrXfica
DATUM: Datum de referencia

TA: Temperatura del aire (ºC)

Unidades y valores especiales:

Horas UTC (Tiempo Universal Coordinado)

Campos incluidos:
Indicativo: Indicativo climatológico
NOMBRE: Nombre estación
ALTITUD: Altitud de la estación (metros)
NOM\_PROV: Provincia
LATITUD: Latitud geográfica
DATUM: Datum de referencia

VV10M: Velocidad media del viento (m/s)
DV10M: Dirección media del viento (º (grados))
VMAX10M: Velocidad mXxima del viento (m/s)
DMAX10M: Dirección de la velocidad máxima del viento (º (grados))



Unidades y valores especiales:

Horas UTC (Tiempo Universal Coordinado)


Como se observa, la insolación hace referencia la radiación solar \textbf{porcentual} y solo esta disponible en Barajas.
La temperatura en cambio solo esta disponible en Cuidad Universitaria y El Retiro.

Queda excluida la base de datos de AEMET.

API publica

Descartada la opción de estación meteorologica, nos inclinamos por una API RESTful publica y gratuita que proporciónase los valores necesarios: Temperatura e Irradiación ademas de Velocidad del viento y Dirección.

Las opciones fueron:
- Madrid AEMET opendata
- El tiempo
- Aqui faltan algunas...
- dark sky weather

[Explicar por que descartamos AEMET opendata y El tiempo]

Finalmente nos quedamos con dark sky weather que era capaz de proporcionar todo.

Los datos necesarios para acceder al API son:

[completar]

Con estos datos ya es posible realizar llamadas en forma de peticiones http y obtener los resultados.

Las funciones usadas fueron:

- Get token
- Get weather

[Formatos]

A la hora de consultar el tiempo y las predicciones incluimos varias opciones para obtener resultados mas concisos.

[opciones]

Por lo tanto, el formato de las respuestas seria:

[Copias schemas del server]


\subsection{DB} 
\label{sub:DB}

La base de datos se ha elegido como no relacional porque...[explicar otra vez]

\subsection{Server} 
\label{sub:server}

Habiendo elegido la fuente de datos, es necesario automatizar la recogida y su almacenado.

Para ello se opto por una aplicación creada con el framework nodejs y una base de datos mongo.

Nodejs porque es versatil y comodo de tratar javascript con el.
MongoDB porque no es necesaria ningun tipo de estructura relacional en los datos.

El servidor se encarga de realizar periodicamente las llamadas a la API de dark sky, anadir la fecha de la llamada (por claridad, ya que los las fechas estan representadas segun la base de tiempo POSIX*), quitar las predicciones horarias excepto del principio del dia e insertar las respuestas en la base de datos.

La arquitectura esta montada en un Ubuntu Desktop 16.04 y dockerizada que permiten desacoplar el software del hardware y proporcionan resiliencia*

Finalmente el servidor esta emplazado fisicamente en [la facultad de informatica(?)] 



\subsection{Extractor de datos} 
\label{sub:extractor_de_datos}

Una vez tenemos la base de datos con contenido sufuciente en continuo crecimiento, ya se puede proceder a extraer los datos de la base de datos y pasarselos al modelo de planta fotovoltaica de simulink para que proporcione una potencia de salida.

Para ello, realizaremos una consulta a la base de datos y escribiremos los resultados en un archivo de texto que reconozca matlab y facilite su manipulación.

[Codigo explicado]


\subsection{Matlab}
\label{sub:Matlab} 

Para la implementación de los modelos se ha optado por usar matlab ya que a diferencia de C++ u otros lenguajes, tiene un toolbox de disenyo de redes neuronales muy completo y practico.

Lista de toolbox usados:
- 




\subsection{Simulink} 
\label{sub:Simulink}

El siguiente modelo simula una planta fotovoltaica de 100kW. 

\begin{figure}[h]
    \includegraphics[width=\textwidth]{Ppv_diagram.png}
    \caption{Diagrama de la planta fotovoltaica de 100kW conectada a una red}
    \label{fig:Ppv_diagram}
\end{figure}


El modelo simula una planta fotovoltaica con las siguientes características:
\begin{itemize}
    \item f
\end{itemize}

Para llevar a cabo la simulación es necesario introducir las variables mas relevantes para el panel fotovoltaico: Radiación y Temperatura. Tiene como resultado la potencia instantanea en watios. El modelo evalua las entradas 1000 veces por segundo dando como resultado una detallada grafica de potencia producida.

La importación y exportación de los valores se ha realizado desde el workspace, donde se pre y post procesan las señales resultantes: Las entradas (Radiación y Temperatura) se formatean de acuerdo a las entradas del modelo y las salidas (La potencia) se discretiza a valor por hora, ya que se evalua la potencia a un ratio de 1000 veces por 1 segundo de simulación.

La quicena de potencia que usaremos es la descrita entre el hl{FECHAS AQUI}  y el eso. Se puede observar en la gráfica XX donde el eje x son los dias y el y la potencia generada en watios.

El resultado de la simulación puede observarse en la siguiente figura:

\begin{figure}[h]
    \includegraphics[width=\textwidth]{Model_cur_outputs_010417.jpg}
    \caption{Salida de la planta fotovoltaica}
    \label{fig:Ppv_output}
\end{figure}

\subsection{Modelos}
\label{sub:Modelos}

Para el estudio se han probado los siguientes modelos.

\begin{itemize}
    \item EWMA
    \item Predictor for adaptive management from ETHZ
    \item 2D linear predictor
    \item Neural network
    \item WCMA
    \item WCMA-PDR
\end{itemize}

Para la experimentación se ha usado el mismo periodo de 15 dias, entre el 15 y el 30 de marzo del 2017.

Los resultados mostrados a continuación se presentan como energía real producida más energía predicha y error entre ambas.


\subsubsection{EWMA}
\label{ssub:ewma}

En la gráfica se puede observar que usa los valores del dia anterior con una ligera atenuacion de la media de las predicciones pasadas. El nivel de error es alto durante el amanecer y el ocaso.

Horizonte de predicción: 24h
$\alpha = TODO$

\begin{figure}[h]
    \includegraphics[width=\textwidth]{EWMA.jpg}
    \caption{EWMA Prediction accurancy}
    \label{fig:ewma_comp}
\end{figure}

\begin{figure}[h]
    \includegraphics[width=\textwidth]{EWMA_error.jpg}
    \caption{EWMA Prediction error}
    \label{fig:ewma_error}
\end{figure}


\subsubsection{Adaptive power management (del ETHZ)}
 \label{ssub:adaptive_power_management}

El resultado de este tipo de predictor es muy similar al EWMA ya que realiza un cálculo basculado de el valor del dia anterior, pero sutilmente mejorado ya que tambien añade a la predicción los valores inmediatamente anteriores. El error en este caso es mas pronunciado en los extremos del dia pero mas estable en los puntos de irradiación estable.

Horizonte de predicción: 24h
$\alpha = TODO$

\begin{figure}[h]
    \includegraphics[width=\textwidth]{ETHZ.jpg}
    \caption{ETHZ Prediction accurancy}
    \label{fig:ethz_comp}
\end{figure}

\begin{figure}[h]
    \includegraphics[width=\textwidth]{ETHZ_error.jpg}
    \caption{ETHZ Prediction error}
    \label{fig:ethz_error}
\end{figure}


\subsubsection{Optimal 2D prediction filter} 
\label{ssub:optimal_2d_prediction_filter}



\begin{figure}[h]
    \includegraphics[width=\textwidth]{Optimal2D.jpg}
    \caption{2D filter Prediction accurancy}
    \label{fig:o2d_comp}
\end{figure}

\begin{figure}[h]
    \includegraphics[width=\textwidth]{Optimal2D_error.jpg}
    \caption{2D filter Prediction error}
    \label{fig:o2d_error}
\end{figure}


\subsubsection{WCMA} 
\label{ssub:wcma}
\begin{figure}[h]
    \includegraphics[width=\textwidth]{WCMA.jpg}
    \caption{WCMA Prediction accurancy}
    \label{fig:wcma_comp}
\end{figure}

\begin{figure}[h]
    \includegraphics[width=\textwidth]{WCMA_error.jpg}
    \caption{WCMA Prediction error}
    \label{fig:wcma_error}
\end{figure}


\subsubsection{WCMA with PDR} 
\label{ssub:wcma_with_pdr}
\begin{figure}[h]
    \includegraphics[width=\textwidth]{WCMA-PDR.jpg}
    \caption{WCMA with PDR Prediction accurancy}
    \label{fig:wcmapdr_comp}
\end{figure}

\begin{figure}[h]
    \includegraphics[width=\textwidth]{WCMA-PDR_error.jpg}
    \caption{WCMA with PDR Prediction error}
    \label{fig:wcmapdr_error}
\end{figure}


\subsubsection{Red neuronal artificial} 
\label{ssub:nn}

El resultado de la red neuronal es especialmente bueno si consideramos valores estables, pero el tiempo de procesado y aprendizaje (El cual deberá ser continuo)

\begin{figure}[h]
    \includegraphics[width=\textwidth]{nn.jpg}
    \caption{neural network Prediction accurancy}
    \label{fig:nn_comp}
\end{figure}

\begin{figure}[h]
    \includegraphics[width=\textwidth]{nn_error.jpg}
    \caption{neural network Prediction error}
    \label{fig:nn_error}
\end{figure}



% Results
\chapter{Resultados} 
\label{cha:resultados}



\section{Solar: Modelos}

Para el estudio de predicción de energía solar se han probado los siguientes modelos.

\begin{itemize}
    \item EWMA
    \item Predictor for adaptive management from ETHZ
    \item 2D linear predictor
    \item WCMA
    \item WCMA-PDR
    \item Neural network
    \item N4SID
\end{itemize}

Para la experimentación se ha usado el mismo periodo de 15 dias, entre el 15 y el 30 de marzo del 2017.

Los resultados mostrados a continuación se presentan como energía real producida más energía predicha y error entre ambas.


\subsubsection{EWMA}
\label{ssub:ewma}

En la gráfica se puede observar que usa los valores del dia anterior con una ligera atenuacion de la media de las predicciones pasadas. El nivel de error es alto durante el amanecer y el ocaso.

Horizonte de predicción: 24h
$\alpha = TODO$

\begin{figure}[h]
    \includegraphics[width=\textwidth]{EWMA.jpg}
    \caption{EWMA Prediction accurancy}
    \label{fig:ewma_comp}
\end{figure}

\begin{figure}[h]
    \includegraphics[width=\textwidth]{EWMA_error.jpg}
    \caption{EWMA Prediction error}
    \label{fig:ewma_error}
\end{figure}


\subsubsection{Adaptive power management (del ETHZ)}
 \label{ssub:adaptive_power_management}

El resultado de este tipo de predictor es muy similar al EWMA ya que realiza un cálculo basculado de el valor del dia anterior, pero sutilmente mejorado ya que tambien añade a la predicción los valores inmediatamente anteriores. El error en este caso es mas pronunciado en los extremos del dia pero mas estable en los puntos de irradiación estable.

Horizonte de predicción: 24h
$\alpha = TODO$

\begin{figure}[h]
    \includegraphics[width=\textwidth]{ETHZ.jpg}
    \caption{ETHZ Prediction accurancy}
    \label{fig:ethz_comp}
\end{figure}

\begin{figure}[h]
    \includegraphics[width=\textwidth]{ETHZ_error.jpg}
    \caption{ETHZ Prediction error}
    \label{fig:ethz_error}
\end{figure}


\subsubsection{Optimal 2D prediction filter} 
\label{ssub:optimal_2d_prediction_filter}

\begin{figure}[h]
    \includegraphics[width=\textwidth]{Optimal2D.jpg}
    \caption{2D filter Prediction accurancy}
    \label{fig:o2d_comp}
\end{figure}

\begin{figure}[h]
    \includegraphics[width=\textwidth]{Optimal2D_error.jpg}
    \caption{2D filter Prediction error}
    \label{fig:o2d_error}
\end{figure}


\subsubsection{WCMA} 
\label{ssub:wcma}

\begin{figure}[h]
    \includegraphics[width=\textwidth]{WCMA.jpg}
    \caption{WCMA Prediction accurancy}
    \label{fig:wcma_comp}
\end{figure}

\begin{figure}[h]
    \includegraphics[width=\textwidth]{WCMA_error.jpg}
    \caption{WCMA Prediction error}
    \label{fig:wcma_error}
\end{figure}


\subsubsection{WCMA with PDR} 
\label{ssub:wcma_with_pdr}

\begin{figure}[h]
    \includegraphics[width=\textwidth]{WCMA-PDR.jpg}
    \caption{WCMA with PDR Prediction accurancy}
    \label{fig:wcmapdr_comp}
\end{figure}

\begin{figure}[h]
    \includegraphics[width=\textwidth]{WCMA-PDR_error.jpg}
    \caption{WCMA with PDR Prediction error}
    \label{fig:wcmapdr_error}
\end{figure}


\subsubsection{Red neuronal artificial} 
\label{ssub:nn}

El resultado de la red neuronal es especialmente bueno si consideramos valores estables, pero el tiempo de procesado y aprendizaje (El cual deberá ser continuo)

\begin{figure}[h]
    \includegraphics[width=\textwidth]{nn.jpg}
    \caption{neural network Prediction accurancy}
    \label{fig:nn_comp}
\end{figure}

\begin{figure}[h]
    \includegraphics[width=\textwidth]{nn_error.jpg}
    \caption{neural network Prediction error}
    \label{fig:nn_error}
\end{figure}


\subsubsection{N4SID} 
\label{ssub:n4sid}

\begin{figure}[h]
    \includegraphics[width=\textwidth]{n4sid_solar.jpg}
    \caption{N4SID Prediction accurancy}
\end{figure}




\section{Eólica: Modelos}
\label{sub:Modelos}

Para el estudio de predicción de energía eólica se han probado los siguientes modelos.

\begin{itemize}
    \item N4SID
    \item ARIMA/ARMA
\end{itemize}

\subsubsection{Eólica: N4SID} % (fold)
\label{ssub:eólica_n4sid}


\begin{figure}[h]
    \includegraphics[width=\textwidth]{n4sid_eolica.jpg}
    \caption{N4SID Prediction accurancy}
\end{figure}

\begin{figure}[h]
    \includegraphics[width=\textwidth]{n4sid_eolica_error.jpg}
    \caption{N4SID Prediction error}
\end{figure}


\subsubsection{Eólica: ARIMA/ARMA} % (fold)
\label{ssub:eólica_arima_arma}

\begin{figure}[h]
    \includegraphics[width=\textwidth]{arma_partial_autocorrelation.jpg}
    \caption{N4SID Prediction accurancy}
\end{figure}
\begin{figure}[h]
    \includegraphics[width=\textwidth]{arma_differenciation.jpg}
    \caption{N4SID Prediction accurancy}
\end{figure}
\begin{figure}[h]
    \includegraphics[width=\textwidth]{arma_train_and_predictionOverTrain.jpg}
    \caption{N4SID Prediction accurancy}
\end{figure}
\begin{figure}[h]
    \includegraphics[width=\textwidth]{arma_test_and_predictionOverTest.jpg}
    \caption{N4SID Prediction accurancy}
\end{figure}

% Closing
\chapter{Conclusiones} % (fold)
\label{cha:resultados}

%%% Bibliografía %%%
\begin{thebibliography}{99}
\addcontentsline{toc}{chapter}{Bibliografía}
\markboth{Bibliografía}{Bibliografía}

\bibitem{Marina} \textsc{Marina Zapater Ata Turk Jose M. Moya Jose L. Ayala Ayse K. Coskun}
 \textit{Dynamic Workload and Cooling Management in High-Efficiency Data Centers}

\bibitem{Risco2016} \textsc{José L. Risco y Saurabh Mital}, \textit{``xDEVS r20150903"} : {\small \url{https://docs.google.com/document/d/1_AMjvXbaUxICmLjbnIUzdrgtITLUV8p1VDKvb8OYS-Y/}}

\end{thebibliography}

\input{chapter-ack}
%% AUTORIZACIÓN %%
\chapter*{Autorización de difusión}
\addcontentsline{toc}{chapter}{Autorización de difusión}
\markboth{Autorización de difusión}{Autorización de difusión}
\section*{Autorización para la difusión del Trabajo Fin de Grado y su depósito en el Repositorio Institucional E-Prints Complutense}
{
\setlength{\parindent}{0cm}
Los abajo firmantes, alumno y tutores del Trabajo Fin de Grado (TFG) en Ingeniería de Computadores de la Facultad de Informática, autorizan a la Universidad Complutense de Madrid (UCM) a difundir y utilizar con fines académicos, no comerciales y mencionando expresamente a su autor, el Trabajo Fin de Grado (TFG) cuyos datos se detallan a continuación. Así mismo autorizan a la Universidad Complutense de Madrid a que sea depositado en acceso abierto en el repositorio institucional con el objeto de incrementar la difusión, uso e impacto del TFG en Internet y garantizar su preservación y acceso a largo plazo.

\bigskip
\bigskip

TÍTULO del TFG: \textbf{\mytitle}.\medskip

Curso académico: \mycourse\\

\bigskip

Nombre del Alumno: \textbf{\myauthor}.\bigskip

Tutores del TFG: \textbf{\myadvisorA~y \myadvisorB}, Departamento de Arquitectura de Computadores y Automática.\\[1.5cm]

\bigskip

\begin{tabular*}{1\textwidth}{@{\extracolsep{\fill}} l r}
\ \ \ \ Fdo.: \myauthor   &                         \\[2.5cm]
\ \ \ \ Fdo.: \myadvisorA & Fdo.: \myadvisorB \ \ \ \\
\end{tabular*}
}


\end{document}